\documentclass[12pt,a4paper]{article}

\usepackage{amsmath}
\usepackage{amsthm}
\usepackage{amssymb}
\usepackage{booktabs}
\usepackage{graphicx}
\usepackage{color}
\usepackage{fullpage}
\usepackage{pdfpages}
\usepackage{listings}
\usepackage{color}
\usepackage{url}
\usepackage{multirow}
% 為了讓表格緊密
\usepackage{float}
% 為了首行空格
%\usepackage{indentfirst}
% % 為表格添加 footnote
% \usepackage{tablefootnote}

% for Chinese
\usepackage{fontspec} % 加這個就可以設定字體
\usepackage[BoldFont, SlantFont]{xeCJK} % 讓中英文字體分開設置
\setCJKmainfont{TW-Sung} % 設定中文為系統上的字型,而英文不去更動,使用原TeX\字型
\renewcommand{\baselinestretch}{1.4}

\parskip=5pt
\parindent=24pt
\newtheorem{lemma}{Lemma}
\newtheorem{ques}{Question}
\newtheorem{prop}{Proposition}
\newtheorem{defn}{Definition}
\newtheorem{rmk}{Remark}
\newtheorem{note}{Note}
\newtheorem{eg}{Example}
\newtheorem{aspt}{Assumption}

\definecolor{emphOrange}{RGB}{247, 80, 0}
\definecolor{stringGray}{RGB}{109, 109, 109}
\definecolor{commentGreen}{RGB}{0, 96, 0}
\definecolor{mygreen}{rgb}{0,0.6,0}
\definecolor{mygray}{rgb}{0.5,0.5,0.5}
\definecolor{mymauve}{rgb}{0.58,0,0.82}

\lstset{
  belowcaptionskip=1\baselineskip,
  breaklines=true,
%   frame=L,
%  xleftmargin=\parindent,
  language = SQL,
  showstringspaces=false,
  basicstyle = \ttfamily, 
  keywordstyle = \bfseries\color{blue}, 
  emph = {symbol1, symbol2},
  emphstyle = \color{red},
  emph = {[2]symbol3, symbol4},
  emphstyle = {[2]\color{emphOrange}},
  commentstyle = \color{commentGreen}, 
  stringstyle = \color{stringGray}, 
%  backgroundcolor = \color{white}, 
%  numbers = left, % 沒有行號,複製貼上測試程式會比較方便
%  numberstyle = \normalsize, 
%	stepnumber = 1, 
%  numbersep = 10pt, 
%  title = ,
}

% "摘要", "表", "圖", "參考文獻"
\renewcommand{\abstractname}{\bf 摘要}
\renewcommand{\tablename}{表}
\renewcommand{\figurename}{圖}
\renewcommand{\refname}{\bf 參考文獻}
\renewcommand{\labelenumii}{\alph{enumii}.}


\begin{document}

\section*{註冊}
\begin{lstlisting}
    SQL
\end{lstlisting}

\section*{登入}
\begin{lstlisting}
    SQL
\end{lstlisting}

\section*{修改密碼}
\begin{lstlisting}
    SQL
\end{lstlisting}

\section*{匿名登入}
\begin{lstlisting}
    SQL
\end{lstlisting}

\section*{貼文}
\begin{lstlisting}
    SQL
\end{lstlisting}

\section*{通知}
\begin{lstlisting}
    SQL
\end{lstlisting}

\section*{搜尋}
\begin{lstlisting}
    SQL
\end{lstlisting}

\section*{檢視排序(留言數)}
\begin{lstlisting}
    SQL
\end{lstlisting}

\section*{檢視排序(時間序)}
\begin{lstlisting}
    SQL
\end{lstlisting}

\section*{檢視(個人主頁)}
\begin{lstlisting}
    SQL
\end{lstlisting}

\section*{檢視(貼文)}
\begin{lstlisting}
    SQL
\end{lstlisting}

\section*{檢視(留言)}
\begin{lstlisting}
    SQL
\end{lstlisting}

\section*{檢視(私訊、通知)}
\begin{lstlisting}
    SQL
\end{lstlisting}

\section*{刪除(貼文)}
\begin{lstlisting}
    SQL
\end{lstlisting}

\section*{刪除(留言)}
\begin{lstlisting}
    SQL
\end{lstlisting}

\section*{私訊}
\begin{lstlisting}
    SQL
\end{lstlisting}

\section*{回饋}
\begin{lstlisting}
    SQL
\end{lstlisting}

\section*{檢視(回饋)}
\begin{lstlisting}
    SQL
\end{lstlisting}

\section*{留言}
\begin{lstlisting}
    SQL
\end{lstlisting}

\end{document}